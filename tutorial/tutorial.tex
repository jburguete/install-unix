\documentclass[a4paper]{article}

\usepackage{graphicx}
\usepackage{color}
\usepackage{listings}

\newcommand{\FIG}[2]
{
	\begin{figure}[ht!]
	\centering
	\includegraphics[scale=#1]{#2}
	\end{figure}
}
\newcommand{\FIGURE}[1]{\FIG{0.35}{#1}}
\newcommand{\FIGUREB}[1]{\FIG{0.26}{#1}}

\newcommand{\RED}[1] {\textcolor{red}{#1}}

\title{Setting up a MSYS2 terminal with some useful compilers, libraries
and tools on Microsoft Windows systems step by step}

\author{Javier Burguete Tolosa}

\date{\today}

\begin{document}

\maketitle

\tableofcontents

\section*{}

SORRY! I have only a spanish license of Windows 7\footnote{Windows 7 is a
registered trademark of Microsoft Corporation}. Therefore, some figures have
spanish texts.

\clearpage

\section{Installing MSYS2}

MSYS2 is a minimal Unix terminal emulator with a software distribution running
over Microsoft Windows systems. In this section we install and we do a direct
access on the desktop.

\begin{enumerate}

\item Download MSYS2 installer (file \RED{msys2-x86\_64-20150202.exe}) on
\newline\RED{http://sourceforge.net/projects/msys2}

\FIGURE{MSYS2-1.png.eps}

\item Click on \RED{Save file} button
\FIGURE{MSYS2-2.png.eps}

\clearpage

\item Click on the downloaded \RED{msys2-x86\_64-20150202.exe} installer
\FIGURE{MSYS2-3.png.eps}

\item Click on the \RED{Execute} button
\FIGURE{MSYS2-4.png.eps}

\clearpage

\item Click on the \RED{Next} button
\FIGURE{MSYS2-5.png.eps}

\item Click on the \RED{Next} button
\FIGURE{MSYS2-6.png.eps}

\clearpage

\item Click on the \RED{Next} button
\FIGURE{MSYS2-7.png.eps}

\item Click on the \RED{Finish} button
\FIGURE{MSYS2-8.png.eps}

\clearpage

\item Create a direct access of the
\RED{C:$\backslash$msys64$\backslash$mingw64\_shell.bat} file sending it to the
desktop
\FIGURE{MSYS2-10.png.eps}

\item Select the new direct access and clicking on the right button and click on
the \RED{Properties} menu option
\FIGURE{MSYS2-12.png.eps}

\clearpage

\item Click on the \RED{Change icon} button
\FIGURE{MSYS2-13.png.eps}

\item Click on the \RED{Accept} button
\FIGURE{MSYS2-14.png.eps}

\clearpage

\item Click on the \RED{Check} button
\FIGURE{MSYS2-15.png.eps}

\item Select the \RED{C:$\backslash$msys64$\backslash$msys2.ico} file and click
on the \RED{Accept} button
\FIGURE{MSYS2-16.png.eps}

\clearpage

\item Select the icon and click on the \RED{Accept} button
\FIGURE{MSYS2-17.png.eps}

\item Click on the \RED{Accept} button
\FIGURE{MSYS2-18.png.eps}

\clearpage

\item Select the direct access clicking on the right button and click on the
\RED{Rename} menu option. Then change the label (i.e. MSYS2 64 bits)
\FIGURE{MSYS2-19.png.eps}

\item Click on the direct access button and you have a MSYS2 (minimal Unix)
terminal
\FIGURE{MSYS2-21.png.eps}

\clearpage

\item Install the required git and make tools by typing in the terminal:
\begin{lstlisting}[language=bash,basicstyle=\scriptsize]
$ pacman -S make git
\end{lstlisting}
and pressing Y in all questions.
\FIGURE{MSYS2-23.png.eps}

\item Install the tools by typing in the terminal:
\begin{lstlisting}[language=bash,basicstyle=\scriptsize]
$ git clone https://github.com/jburguete/install-unix.git
$ cd install-unix
$ make
\end{lstlisting}
and pressing yes or Y in all questions.
\FIGURE{MSYS2-24.png.eps}

\end{enumerate}

\clearpage

\section{Installing MiKTeX LaTeX (optional)}

MiKTeX is a distribution of the LaTeX text processor for Windows. We install
here the basic 64 bits distribution.

\begin{enumerate}

\item Download the distribution from \RED{http://miktex.org/download} clicking
on the \RED{Other Downloads} panel and the
\RED{Basic MiKTeX 2.9.4813 64-bit Installer} (the number is the version at the
date to make this tutorial)
\FIGUREB{Latex-1.png.eps}

\clearpage

\item Click on the \RED{Save file} button
\FIGUREB{Latex-2.png.eps}

\item Click on the downloaded installer (\RED{basic-miktex-2.9.4813-x64.exe}
is the last version at the date to make this tutorial)
\FIGUREB{Latex-3.png.eps}

\clearpage

\item Click on the \RED{Yes} button
\FIGUREB{Latex-4.png.eps}

\item Accept the license and click on the \RED{Next} button
\FIGUREB{Latex-5.png.eps}

\clearpage

\item Select the \RED{Anyone who uses the computer (all users)} option and click
on the \RED{Next} button
\FIGUREB{Latex-6.png.eps}

\item Select the installation directory and click on the \RED{Next} button
\FIGUREB{Latex-7.png.eps}

\clearpage

\item Select the prefered options and click on the \RED{Next} button
\FIGUREB{Latex-8.png.eps}

\item Click on the \RED{Start} button
\FIGUREB{Latex-9.png.eps}

\clearpage

\item Click on the \RED{Next} button
\FIGUREB{Latex-10.png.eps}

\item Click on the \RED{Close} button
\FIGUREB{Latex-11.png.eps}

\end{enumerate}

\end{document}

\clearpage

\section{Installing Eclipse (optional)}
